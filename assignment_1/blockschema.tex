\documentclass{article}

\usepackage{tikz}
\usetikzlibrary{shapes,arrows}
\begin{document}


\tikzstyle{block} = [draw, rectangle, 
    minimum height=3em, minimum width=3em]
\tikzstyle{sum} = [draw, circle, node distance=3cm]
\tikzstyle{sum1} = [draw, circle, node distance=3cm, inner sep=0pt]

\tikzstyle{input} = [coordinate]
\tikzstyle{output} = [coordinate]
\tikzstyle{pinstyle} = [pin edge={to-,thin,black}]
\newcommand{\summa}{\huge$\sum_{}^{}$}

% The block diagram code is probably more verbose than necessary
\begin{tikzpicture}[auto, node distance=2cm,>=latex']
    % We start by placing the blocks
    \node [input, name=input] {};
    
    \node [sum, right of=input, ] (sum) {\summa};
    \node [block, right of=sum, ] (F) {F(s)};
    \node [block, right of=F, ] (G) {G};
    \node [sum, right of=G] (system) {\summa};
    \node [block, above of=system] (K) {$K_u$};       
    \node [block, right of=system] (G2) {$G_m (s)$};
    \node [sum, below of= system] (sum2) {$\Omega$} ;
    
    \draw [->] (sum) -- node {} (F);
    \draw [->] (F) -- node[name=u] {$u$} (G);
    \draw [->] (G) -- node[name=gc] {} (system);
    \draw [->] (K) -- node[name=ksys] {} (system);
    \draw [->] (system) -- node[name=gc] {} (G2);
    \node [output, right of=G2] (output) {out};
   
    \draw [->] (input) -- node [name=line] {$\Omega$} (sum);
    \draw [->] (G2) -- node [name=y] {$\Omega$}(output);
    \draw [->] (y) |- node  {}(K);
    \draw [-] (y) |- node  {}(sum2);
    \draw [->]	(sum2) -| node[pos=0.95] {$-$} node [near end]   {}(sum);
    
    

\end{tikzpicture}

\end{document}