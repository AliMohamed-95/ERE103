\documentclass{article}

\usepackage{tikz}
\usetikzlibrary{shapes,arrows}
\begin{document}


\tikzstyle{block} = [draw, rectangle, 
    minimum height=3em, minimum width=3em]
\tikzstyle{sum} = [draw, circle, node distance=3cm]


\tikzstyle{input} = [coordinate]
\tikzstyle{output} = [coordinate]
\tikzstyle{pinstyle} = [pin edge={to-,thin,black}]
\newcommand{\summa}{\huge$\sum_{}^{}$}

% The block diagram code is probably more verbose than necessary
\begin{tikzpicture}[auto, node distance=2cm,>=latex']
    % We start by placing the blocks
    \node [input, name=input] {};
    
    \node [sum, right of=input, ] (summa) {\summa};
    \node [block, right of=summa, ] (F) {F(s)};
    \node [sum, right of=F] (sum2) {\summa};
    \node [block, right of=sum2, ] (G) {$G_e(s)$};
    
    \node [block, above of=sum2] (K) {$K_u$};       
    \node [block, right of=G] (G2) {$G_m (s)$};
    \node [sum, below of= sum2] (omega) {$\Omega$} ;
    \node [block, right of = G2] (integ){$\int$};
    
    \draw [->] (summa) -- node {E} (F);
    \draw [->] (F) -- node[pos=0.95] {$ + $} node [near end] [name=u] {u} (sum2);
    
    \draw [->] (sum2) -- node[name=gc] {} (G);
    \draw [->] (K) -- node [pos=0.95] {$ - $} node [near end][name=ksys] {} (sum2);
    \draw [->] (G) -- node[name=gc] {T} (G2);
    \node [output, right of=integ] (output) {out};
   
    \draw [->] (input) -- node[pos=0.95] {$ + $} node [near end] [name=line] {$\phi_r$} (summa);
     
    
    \draw [->] (G2) -- node [name=y] {$\Omega$}(integ);
   \draw [->] (y) |- node  {}(K);
    \draw [->] (integ) -- node [name=outl] {$\phi$}(output);
   \draw [-] (outl) |- node  {}(omega);
    \draw [->]	(omega) -| node[pos=0.95] {$-$} node [near end]   {}(summa);
    
    

\end{tikzpicture}

\end{document}